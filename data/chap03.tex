% !TeX root = ../sustechthesis-example.tex

\chapter[离子阱量子计算系统]{离子阱量子计算系统}

\textcolor{red}{
这部分简单讲一下离子阱量子计算的发展历史,重点介绍离子阱量子计算系统的基本组成,不针对特别具体的系统...
}

\section[离子阱的发展]{离子阱的发展}

\textcolor{red}{这部分主要参考文献\cite[p2]{Bruzewicz_Chiaverini_McConnell_Sage_2019}}






\section[囚禁离子的光场耦合]{囚禁离子的光场耦合}
\textcolor{red}{这部分主要参考文献\cite[p3-8]{Leibfried_Blatt_Monroe_Wineland_2003}}
\subsection[离子阱的囚禁原理]{离子阱的囚禁原理}

% \subsection[囚禁离子的光场耦合]{囚禁离子的光场耦合}
\subsection[二能级近似]{二能级近似}

\subsection[耦合的理论表述]{耦合的理论表述}

\subsection[离子内态测量]{离子内态测量}


\subsection[离子运动态测量]{离子运动态测量}














\section[离子的激光冷却]{离子的激光冷却}
\textcolor{red}{主要介绍原理,不讲太多实验细节和结果,\textcolor{red}{这部分主要参考文献\cite[p16-19]{Leibfried_Blatt_Monroe_Wineland_2003}}}
\subsection[多普勒冷却]{多普勒冷却}

\subsection[边带冷却]{边带冷却}



\section[Mølmer-Sørensen门]{Mølmer-Sørensen门}
\textcolor{red}{这部分主要参考文献\cite[]{Azuma_2023}}





\section[系统的组成——离子阱系统]{系统的组成——离子阱系统}
\textcolor{red}{这部分尽量结合实验室的仪器设备展开叙述}

\subsection[囚禁电极]{囚禁电极}


\subsection[微波信号]{微波信号}


\subsection[真空系统]{真空系统}


\subsection[螺线管谐振腔]{螺线管谐振腔}











\section[系统的组成——光学系统]{系统的组成——光学系统}
\textcolor{red}{这部分尽量结合实验室的仪器设备展开叙述}

\subsection[冷却激光]{冷却激光}

\subsection[操控激光]{操控激光}






\section[系统的组成——测控系统]{系统的组成——测控系统}
\textcolor{red}{这部分尽量结合实验室的仪器设备展开叙述}
\subsection[测控系统的构架]{测控系统的构架}

