% !TeX root = ../sustechthesis-example.tex

\chapter[离子阱量子计算系统]{离子阱量子计算系统\label{section:ion_trap_quantum_computation_system}}

\textcolor{red}{
这部分简单讲一下离子阱量子计算的发展历史,重点介绍离子阱量子计算系统的基本组成,不针对特别具体的系统...
}

\section[离子阱的发展]{离子阱的发展}

\textcolor{red}{这部分主要参考文献\cite[p2]{Bruzewicz_Chiaverini_McConnell_Sage_2019}}




% \section[离子阱的囚禁]{离子阱的囚禁}

\section[离子在RF阱中的经典运动]{离子在RF阱中的经典运动\label{section:ion_classical_motion}}
% \textcolor{red}{主要参考文献\cite[chap2-A]{Leibfried_Blatt_Monroe_Wineland_2003}}

离子阱采用射频阱对离子进行动态囚禁,最具代表性的一类是四极阱,它及它的一些变型也是目前在离子阱量子计算研究中应用最广泛的一类离子阱。四极阱的电势描述如下:
\begin{align}
    % \label{eq:quadrupolar_trap_potential}
    \Phi(x, y, z, t) = &U\frac{1}{2}(\alpha x^2 + \beta y^2 + \gamma z^2) \label{eq:time_independent_part}\\
    \label{eq:time_dependent_part}
    &+ \tilde{U}\cos (\omega_{rf}t)\frac{1}{2}(\alpha ' x^2 + \beta ' y^2 + \gamma ' z^2) 
\end{align}
其中$\Phi(x, y, z, t)$是四极阱的电势,公式\eqref{eq:time_independent_part}是不依赖时间的项,公式\eqref{eq:time_dependent_part}是随时间变化的项。整个电势的表达式每时每刻都要满足拉普拉斯方程(Laplace equation)$\Delta \Phi=0$的约束条件,从中可以导出整个四极阱的几何参数约束:
\begin{align}
    \alpha + \beta + \gamma =0,\\
    \alpha ' + \beta ' + \gamma ' =0
\end{align}

其中各参数的定义在公式\eqref{eq:time_independent_part}和公式\eqref{eq:time_dependent_part}定义。从这些限制可以明显看出,在自由空间中电势不可能稳定地产生局部三维最小值,因此电势只可能以动态方式来对离子进行囚禁。幸运的是,通过对四极阱几何参数的选择,再结合适当的驱动微波的频率和驱动电压我们可以做到这一点。其中一种几何参数选择如下:
\begin{align}
    -(\alpha + \beta )= \gamma > 0,\\
    \alpha ' = - \beta '
\end{align}
这种几何参数的设置会使离子在$x,y$平面上动态地被囚禁,在$z$方向上静态地被囚禁。在这种设置的离子阱中,多个离子会沿着$z$轴形成线性的离子链,这便是人们所知的\emph{线性离子阱(Linear Trap)},也被称为\emph{Paul Trap}\cite[]{Paul_1990}。在接下来的两小节里我将介绍囚禁离子的经典运动方程及其解析解(第\ref{section:classical_motion}节),并给出这些解的低阶近似(第\ref{section:lowest_order_approximation}节)。

\subsection[经典运动方程]{经典运动方程\label{section:classical_motion}}

一个质量为$m$电荷量为$Z|e|$的粒子在如公式\eqref{eq:time_independent_part}所描述的电场中的经典运动方程由Paul等人\cite[p415]{Paul1958}给出。粒子的运动在空间坐标方向上中是解耦的。下面只讨论$x$方向上的运动;其他方向可以类似地处理。运动方程如下:
\begin{align}
    \ddot{x}=-\frac{Z|e|}{m}\frac{\partial \Phi}{\partial x}=-\frac{Z|e|}{m}[U\alpha + \tilde{U}\cos(\omega_{rf}t)\alpha ']x
\end{align}

经过下面的参数代换,这个方程可以转化为标准的\emph{马修方程(Mathieu Equation, ME)}形式:
\begin{align}
    \frac{d^2x}{d\xi^2}+[a_x-2q_x\cos(2\xi)]x=0\label{eq:mathieu_equation}
\end{align}

相应的参数代换为:
\begin{align}
    \xi=\frac{\omega_{rf}t}{2},\ a_x=\frac{4Z|e|U\alpha}{m\omega_{rf}^2},\ q_x=\frac{2Z|e|\tilde{U}\alpha '}{m\omega_{rf}^2}\label{eq:parameters_substitution}
\end{align}

ME方程属于一般的周期系数微分方程。它的稳定解的一般形式可以由\emph{弗洛奎定理(Floquet Theorem)}导出\cite[]{McLachlan, McQuarrie}:
\begin{align}
    x(\xi)=&Ae^{i\beta_x\xi}\sum_{n=-\infty}^{\infty}C_{2n}e^{i2n\xi}\\
    &+ Be^{-i\beta_x\xi}\sum_{n=-\infty}^{\infty}C_{2n}e^{-i2n\xi}\label{eq:mathieu_solution}
\end{align}

其中实值特征指数$\beta_x$和系数$C_{2n}$仅是$a_x$和$q_x$的函数,不依赖于初始条件。$A$和$B$是任意常数,可用于满足边界条件或规格化特解。将公式\eqref{eq:mathieu_solution}代入公式\eqref{eq:mathieu_equation}可以得到一个递归关系:
\begin{align}
    C_{2n+2}-D_{2n}C_{2n}+C_{2n-2}=0,\\
    D_{2n}=[a_x-(2n+\beta_x)^2]/q_x\label{eq:recursion_raltion}
\end{align}

这一递归关系将实值特征指数$\beta_x$、系数$C_{2n}$与$a_x$、$q_x$联系起来。通过进一步地整理也可以得到$C_{2n}$的表达式:
\begin{align}
    C_{2n+2}=\frac{C_{2n}}{D_{2n}-\frac{1}{D_{2n+2}-\frac{1}{\dots}}}\\
    C_{2n}=\frac{C_{2n-2}}{D_{2n}-\frac{1}{D_{2n-2}-\frac{1}{\dots}}}\label{eq:c_2n_fraction}
\end{align}

利用结合上述公式$\beta_x$也可以计算:
\begin{align}
    \beta_x^2=a_x-q_x\left(\frac{1}{D_0-\frac{1}{D_2-\frac{1}{\dots}}} + \frac{1}{D_0-\frac{1}{D_{-2}-\frac{1}{\dots}}}\right) \label{eq:beta_x_fraction}
\end{align}

可以根据所需的精度,选择截断公式\eqref{eq:c_2n_fraction}和公式\eqref{eq:beta_x_fraction}中的连分式来获取相应的结果。
实际上,对于实验中常用到的典型$a_x$和$q_x$值,连分式中高阶项的贡献会迅速下降。

\textcolor{red}{这里的$a_x, q_x$具体的含义后面有时间了可以再看看。}
\subsection[低阶近似]{低阶近似\label{section:lowest_order_approximation}}
实际实验系统中采用的在公式\eqref{eq:parameters_substitution}中定义的参数往往是满足$(|a_x|,q_x^2)\ll 1$的。在此条件下,假设$C_{\pm 4}\simeq 0$,则可以得到$x(t)$轨迹的\emph{低阶近似(Lowest-order Approximation)}。再同时设置初始条件$A=B$,公式\ref{eq:recursion_raltion}可以得到:
\begin{align}
    \beta_x\approx \sqrt{a_x+q_x^2/2},\\
    x(t)\approx2AC_0\cos\left(\beta_x\frac{\omega_{rf}}{2}t\right)\left[1-\frac{q_x}{2}\cos(\omega_{rf}t)\right]\label{eq:classical_motion_solution}
\end{align}

囚禁离子在$x$方向上的轨迹$x(t)$的由频率为$\nu=\beta_x\omega_{rf}/2\ll \omega_{rf}$的谐波振荡叠加频率为$\omega_{rf}$的RF频率造成的\emph{驱动位移}组成,分别称为\emph{长期运动(Secular-motion)}和\emph{微运动(Micro-motion)}两者相位相差$180^\circ$;
离子在离子阱中的微运动的频率为$\omega_{rf}\ll \nu$,且其振幅为长期运动振幅的$q_x/2\ll 1$,这也是它被称为微运动的原因。如果忽略微运动,则长期运动可以近似为频率为$\nu$的谐振子的运动。在大多数情况下,如果离子处于相当低的动能,即使我们用量子力学的方法来处理离子的质心运动,这个处理也是合理的。


\section[离子在RF阱中的量子力学运动]{离子在RF阱中的量子力学运动\label{section:quantum_motion}}
% \textcolor{red}{主要参考文献\cite[chap2-B]{Leibfried_Blatt_Monroe_Wineland_2003}}

如第\ref{section:ion_classical_motion}节中已经讨论过的,经典中的运动分析似乎已经可以很好地描述离子在离子阱中的运动了。但是,由于四极阱产生的囚禁势场不是静态的而是与时间相关的,因此不能理所当然地认为在有效时间平均势中量化运动已经为我们提供了囚禁离子足够的图景。实际上,在离子阱的实验中,即使是对离子陷阱中的冷却过程的简单解释,以及对非经典状态的描述,也都依赖于运动的量子力学图景的。

在接下来的两小节里,我将根据文献\cite[]{Arimondo_Phillips_Strumia_1992}中的方法导出囚禁离子在射频场中的量子力学表述。同时在这里讨论,在实验中使用的捕获参数范围内,囚禁离子的量子化运动可以用静态谐振子来近似。

\subsection[量子力学运动方程]{量子力学运动方程}
对于囚禁离子运动的量子力学处理,我们假设与时间相关的势在囚禁离子质心的三个笛卡尔坐标中的每一个中都是二次的(一维谐振子势\cite[]{Solimeno_Di_Porto_Crosignani})。然后,与经典运动一样,问题可分为三个一维问题。在一维中,用各自的算子$\hat{x}$替换坐标$x$,于是可以将与时间相关的势$V(T)$写为:
\begin{align}
    V(t)=\frac{m}{2}W(t)\hat{x}^2
\end{align}

其中,
\begin{align}
    W(t)=\frac{\omega_{rf}^2}{4}\left[a_x+2q_x\cos(\omega_{rf}t)\right]
\end{align}

可以被认为是一个时变弹簧常数,它的作用类似于在静态势谐振子中$\omega^2$的作用。在以上的定义下,囚禁离子运动的哈密顿量$H^{(m)}$的形式和我们在量子力学中处理的静态谐振子的哈密顿量很相似:
\begin{align}
    \hat{H}^{(m)}=\frac{\hat{p}^2}{2m}+\frac{m}{2}W(t)\hat{x}^2\label{eq:static_harmiltonian_oscillator}
\end{align}

于是我们可以很轻松地写出这些运动算子在\emph{海森堡图景(Heisenberg Picture)}下的的方程:
\begin{align}
    \dot{\hat{x}}= \frac{1}{i\hbar}\left[\hat{x,\hat{H}^{(m)}}=\frac{\hat{p}}{m}\right],\\
    \dot{\hat{p}}= \frac{1}{i\hbar}\left[\hat{p},\hat{H}^{(m)}\right]=-mW(t)\hat{x}
\end{align}

他们的一个更紧凑的方程形式如下:
\begin{align}
    \ddot{\hat{x}}+W(t)\hat{x}=0 \label{eq:quantum_motion_equation}
\end{align}

如果用函数$u(t)$替换算子$\hat{x}$,我们可以很容易验证这个公式\eqref{eq:quantum_motion_equation}与\emph{马修方程}\eqref{eq:mathieu_equation}是等价的。这就是我们能够借助前面所叙述的马修方程的解来寻找公式\eqref{eq:quantum_motion_equation}的解。添加边界条件:
\begin{align}
    u(0)=1,\ \hat{u}(0)=i\nu \label{eq:boundary_condition}
\end{align}

这对应于公式\eqref{eq:mathieu_solution}中的$A=1,\ B=0$,可以得到:
\begin{align}
    u(t)=e^{i\beta_x\omega_{rf}/2}\sum_{n=-\infty}^{\infty} C_{2n}e^{i n\omega_{rf}t}\equiv e^{i\beta_x\omega_{rf}t/2}\Phi(t) \label{eq:ut_expression}
\end{align}

其中$\Phi(t)$是一个周期为$T=2\pi/\omega_{rf}$的周期函数。于是公式\eqref{eq:boundary_condition}变为:
\begin{align}
    u(0)=\sum_{n=-\infty}^{\infty}C_{2n}=1,\ \nu = \omega_{rf}\sum_{n=-\infty}^{\infty}C_{2n}(\beta_x/2+n)
\end{align}

这个解及其复共轭是线性独立的;因此,它们服从\emph{Wronskian恒等式}:
\begin{align}
    u^*(t)\dot{u}(t)-u(t)\dot{u}^*(t)=u^*(0)\dot{u}(0)-u(0)\dot{u}^*(0)=2 i \mu
\end{align}

未知坐标$\hat{x}(t)$和$u(t)$满足相同的微分方程,因此复杂的线性组合:
\begin{align}
    \hat{C}(t)=\sqrt{\frac{m}{2\hbar \nu}}i\left\{u(t)\dot{\hat{x}(t)-\dot{u}(t)\hat{x}(t)}\right\} \label{eq:complex_combination}
\end{align}

与其如下的Wronskian恒等式成正比,并且在时间上也是恒定的:
\begin{align}
    \hat{C}(t)=\hat{C}(0)=\frac{1}{\sqrt{2m \hbar \nu}}\left[m\nu\hat{x}(0)+i\hat{p}(0)\right]
\end{align}

此外,等式右边恰好是质量$m$和频率$\nu$的在静态谐振子势场中的湮灭算符:
\begin{align}
    \hat{C}(t)=\hat{C}(0)=\hat{a}
\end{align}
也就是说有如下式:
\begin{align}
    \left[\hat{C},\hat{C}^\dagger\right]=\left[\hat{a},\hat{a}^\dagger=1\right]
\end{align}

这个静态势场中的谐振子在后续将被称为\emph{参考谐振子(Reference Oscillator)}。

海森堡算符$\hat{x}(t)$和$\hat{x}(t)$可以用$u(T)$和参考振荡器的算符用公式\eqref{eq:complex_combination}重新表示:

\begin{align}
    \hat{x}(t)=\sqrt{\frac{\hbar}{2m\nu}}\left\{\hat{a}u^*(t)+\hat{a}^\dagger u^(t)\right\},\\
    \hat{p}(t)=\sqrt{\frac{\hbar m}{2\nu}}\left\{\hat{a}\dot{u}^*(t)+\hat{a}^\dagger \dot{u}^(t)\right\}
\end{align}

所以囚禁离子的整个时间依赖性由特殊解$u(t)$及其复共轭给出。
这样一来,对于接下来的计算,在海森堡图景中表述一些列的时间依赖波函数就很方便了。同样,上面使用的参考振荡子也将非常有帮助。与静态势的情况类似,我们将考虑一系列的基态$\ket{n,t}$,其中$n=1,2,\dots,\infty$。这些状态被称为谐波振荡器\emph{数态(Fock States)}的动态对应物。参考振荡子$\ket{n=0}_\nu$的基态满足条件:
\begin{align}
    \hat{a}\ket{n=0}_\nu=\hat{C}(t)\ket{n=0}_\nu=0 \label{eq:obey_condition}
\end{align}

由于海森堡算子$\hat{C}$是通过$\hat{C}(t)=\hat{U}^\dagger(t)\hat{C}_S\hat{U}(t)$与$\hat{C}_S$联系起来的,我们可以很快得到(其中$\hat{U}(t)=\exp{\left[-(i/\hbar)\hat{H}^{(m)}\right]}$):
\begin{align}
    \hat{C}_S(t)\hat{U}(t)\ket{n=0}_\nu=\hat{C}_S(t)\ket{n=0,t}=0 \label{eq:oscillator_condition}
\end{align}

只需要通过将公式\eqref{eq:obey_condition}左侧与$\hat{U}(t)$相乘,并注意到$\hat{U}(t)\ket{n=0}_\nu$是从静态潜在参考振荡子的基态演变而来的时间相关振荡器的薛定谔态。由于薛定谔算子$C_S(t)$的时间依赖完全取决于$u(t)$的时间演化,于是公式\eqref{eq:oscillator_condition}等价于:
\begin{align}
    \left[u(t)\hat{p}-m\dot{u}\hat{x}\right]\ket{n=0,t}=0
\end{align}

在坐标空间表述为:
\begin{align}
    \left\{u(t)\frac{\hbar}{i}\frac{\partial}{\partial x'}\right\}\braket{x'|n=0,t}=0
\end{align}

归一化后的解为:
\begin{align}
    \braket{x'|n=0,t}=\left(\frac{mv}{\pi\hbar}\right)^{1/4}\frac{1}{\{u(t)\}^{1/2}}\exp\left[\frac{i m}{2\hbar}\frac{\dot{u}(t)}{u(t)}x'^2\right]
\end{align}

与静态势谐振子完全类似,可以通过创建算子$\hat{C}_S^\dagger(t)$对基态重复操作来创建完全正交基的所有其它状态:
\begin{align}
    \ket{n,t}=\frac{\left[\hat{C_S^\dagger(t)}\right]^n}{\sqrt{n!}}\ket{n=0,t}\label{eq:basic_sattes}
\end{align}

将$u(t)$如公式\eqref{eq:ut_expression}重写后,在坐标空间表述为:

\begin{align}
    \braket{x'|n,t}=\exp\left[-i\left(n+\frac{1}{2}\right)\nu t\right]\chi_n(t) \label{eq:quantum_states_expression}
\end{align}

其中,$H_n$是$n$阶厄米多项式,$\chi_n(t)$表达式如下:
\begin{align}
    \chi_n(t)=\frac{1}{\sqrt{2^n n!}}\left(\frac{m\nu}{\pi  \hbar}\right)^{1/4}
    \frac{\exp\{-i n \arg\left[\Phi(t)\right]\}}{\{\Phi(t)\}^{1/2}}\\
    \times H_n\left\{\left[\frac{m\nu}{\hbar|\Phi|^2}\right]^{1/2}x'\right\}\\
    \times \exp\left\{\frac{m\nu }{2\hbar}\left[1-\frac{i\Phi(t)}{\nu\Phi(t)}\right]x'^2\right\}
\end{align}

经典的微运动作为射频驱动场周期的脉动出现在波函数中。对于静态势谐振子,能量本征态的演化只将波函数乘以相位因子(这就是为什么它们被称为静止态)。在此处研究的时间相关电势场中,同样如此,不同之处仅在于这里时间只能取得RF周期$T=2\pi/\omega_{rf}$的整数倍。公式\ref{eq:quantum_states_expression}给出的状态并不是能量本征态(它们周期性地与驱动场交换能量,类似于经典的微运动),但它们是时间相关势中可能的平稳状态的很好的近似。因此,它们通常被称为\emph{准平稳状态(Quasistationary States)}。

紧接着的小结节中将介绍与第\ref{section:lowest_order_approximation}节中提出的经典伪势解类似的量子力学中的运动解的低阶近似,找到对静态势谐振子图像的最低阶修正。


\subsection[量子低阶近似]{量子低阶近似}
量子力学中的低阶近似从导出$u(t)$的近似表达开始。与经典的情况类似,低阶近似需要满足条件:$|a_x|,q_x^2\ll 1$、$C_{\pm 4}=0$。结合公式\eqref{eq:boundary_condition}中的初始条件可以得到:
\begin{align}
    \beta_x\approx\sqrt{a_x+q_x^2/2},\ \nu\approx\beta_x\omega_{rf}/2,\\
    u(t)\approx\exp{i\nu t}\frac{1+(q_x/2)\cos(\omega_{rf}t)}{1+q_x/2}\label{eq:quantum_lowest_order_approximation}
\end{align}

这实质上就是前面第\ref{section:lowest_order_approximation}节在公式\eqref{eq:classical_motion_solution}中找到的经典解。
仍然必须强调的是,只有在这种低阶近似中,参考谐振子的频率$\nu$才等于特征指数$\beta_x\omega_{rf}/2$。现在很明显地可以看出$\chi_n(t)$以周期$T_{rf}$进行周期性呼吸。具体可以从基态波函数的近似表达式$\chi_0(t)$中看到:
\begin{align}
    \chi_0(t)=\left(\frac{m\nu}{\pi \hbar}\right)^{1/4}\sqrt{\frac{1+q_x/2}{1+(q_x/2)\cos(\omega_{rf}t)}}
    \times \exp\left(\left\{i\frac{m\omega_{rf}\sin(\omega_{rf}t)}{2\hbar\left[1/q_x+\cos(\omega_{rf}t)\right]}-\frac{m\nu}{2\hbar}\right\}x'^2\right)
\end{align}

而公式\eqref{eq:quantum_states_expression}中的相位因子由基态伪能量$\hbar\nu/2$控制。如果设置$\omega_{rf}=0$,则这个表达式与静态谐波势基态波函数相同。




\section[阱中离子的一些特别的运动量子态]{阱中离子的一些特别的运动量子态}
\textcolor{red}{主要参考文献\cite[chap2-C]{Leibfried_Blatt_Monroe_Wineland_2003}}

接着会讨论一些类似于静态势谐振子的数量和离子阱中一些特殊类别的运动态。这其中有些是非经典的但却让人不禁联想到经典的运动。这部分介绍的运动态都已经被实验观测和验证过了,因此也将介绍如何创建这些运动态。


\subsection[数算子和它的本征态]{数算子和它的本征态}

为了能更好地探索离子阱中的囚禁离子和静态势阱中的谐振子的联系,把运动态表述为以参考谐振子的本征态为基矢的形式将很有帮助。我们首先在海森堡途径中讨论。因为$\hat{C}(t)$是时间独立的,因此这个$\hat{N}$算子也是时间独立的:
\begin{align}
    \hat{N}=\hat{C}^\dagger(t)\hat{C}(t)=\hat{a}^\dagger\hat{a}
\end{align}

它的本征态也是时间独立的,也就是我们在静态势阱中所熟知的\emph{数态或Fock态},它拥有一套相应的升降算符:

\begin{align}
    \hat{a}\ket{n}_\nu=\sqrt{n}\ket{n-1}_\nu,\ \hat{a}^\dagger\ket{n}_\nu=\sqrt{n+1}\ket{n+1}_\dagger,\ \hat{N}\ket{n}_\nu=n\ket{n}_\nu
\end{align}

转换到薛定谔图景后可以得到:
\begin{align}
    \hat{U}^\dagger(t)\hat{N}\hat{U}(t)=\hat{U}^\dagger(t)\hat{C}^\dagger(t)\hat{U}(t) \hat{U}^\dagger(t)\hat{C}(t)\hat{U}(t) = \hat{C}_S^\dagger(t)\hat{C}_S(t)
\end{align}

这个算子的本质态和本征值可以用上节中的公式\eqref{eq:basic_sattes}得到:
\begin{align}
    \hat{C}_S(t)\ket{n,t}=\sqrt{n}\ket{n-1,t},\\
    \hat{C}_S^\dagger(t)\ket{n,t}=\sqrt{n+1}\ket{n+1,t}
\end{align}

如果使:
\begin{align}
    \hat{N}_S(t)\ket{n,t}=n\ket{n,t}
\end{align}

那么这些薛定谔图景下的本征态就可以像静势场谐振子那样使用了,并且静势场中升降算符的所有代数性质都可以对应到$\hat{C}_S(t)$和$\hat{C}_S^\dagger(t)$。
唯一的区别就是这里计算得到的并不是整个系统能量的本征态,因为微运动会周期地改变离子的动能。

任何运动态都可以表示为这些数态的叠加态:
\begin{align}
    \Psi = \sum_{0}^{\infty}c_n\ket{n,t}\label{eq:superposition_states}
\end{align}

这样的态表述会在后续章节的叙述中用到,简单起见,除了专门研究时间相关问题外后面通常会将$\ket{n,t}$简写成$\ket{n}$。






\subsection[相干态]{相干态}

在静势场谐振子中,离子的运动$\ket{\alpha}$的相干态对应于位置表示中的高斯最小不确定性波包,其中心在谐波中经典地振荡并保持其形状。
波包的形状与基态波函数相同。Glauber表明,在动态囚禁电势中从初始相干状态演变的状态也是如公式\eqref{eq:quantum_states_expression}描述的高斯基态的位移形式\cite[]{Arimondo_Phillips_Strumia_1992}。
位移的高斯与基态具有相同的呼吸,但不扩散,其重心遵循陷阱中离子的经典轨迹(现在长期运动和微运动)。
这种情况最先由薛定谔在试图构造反映谐振子经典运动的波包时提出\cite[]{Schrödinger1926}。
\textcolor{red}{这里的 breathing 应该怎么翻译?}

“相干态”的称呼最早是由Glauber提出的\cite[]{Glauber1963Photon,Dewitt_Blandin_Cohen_Tannoudji},与光场的量子态有关。定义相干态的方式有很多种(详见文献\cite[]{Klauder_Skagerstam_1985})。比如,它们是湮灭算子的本征态,相应的本征值为复数$\alpha$:
\begin{align}
    \hat{C}_S(t)\ket{\alpha}=\alpha\ket{\alpha}
\end{align}

这个态很容易用公式\eqref{eq:superposition_states}中的形式表示:
\begin{align}
    c_n=\frac{\alpha^n}{\sqrt{n!}}\exp(-|\alpha|^2/2)
\end{align}

这便是这个算子的本征态,它的数态概率密度分布是\emph{泊松分布(Poissonian Distribution)}:
\begin{align}
    P_n=|c_n|^2=|\braket{n|\alpha}|^2=(\bar{n}^n e^{-\bar{n}})/n!, \bar{n}=|\alpha|^2
\end{align}

另一个流行的方式是将相干态表述为位移算符的动作:
\begin{align}
    \hat{D}(\alpha)=\exp\left[\alpha\hat{C}_S^\dagger(t)-\alpha^* \hat{C}_S(t)\right]
\end{align}

在真空态中,
\begin{align}
    \hat{D}(\alpha)\ket{0}=\ket{\alpha}
\end{align}

连续应用一些列的位移算子的作用结果在相因子上也是相加的:

\begin{align}
    \hat{D}(\alpha)\hat{D}(\beta)=\hat{D}(\alpha+\beta)e^{\alpha\beta^*-\alpha^*\beta}
\end{align}

因此这些位移构成了一个自然单元为$\hat{D}(0)=\hat{I}$的群。
注意由于等式右边的额外相位,使得位移算子通常是不对易的。


\subsection[压缩真空态]{压缩真空态}

\emph{海森堡不确定性关系(Heisenberg Uncertainty Relation)}表明在任何量子态中位置和动量的方差的乘积的下界为$\hbar^2/4$。
静势场中谐振子与所有其它相干态的基态是最小不确定度状态,其中位置的方差为$(\Delta x)^2=\braket{x^2}-\braket{x}^2=1/(m\nu)\hbar/2$;动量的方差为$(\Delta p)^2=\left(m\nu\right)\hbar/2$。

如果现在我们“挤压”位置方差,那么动量方差必须变宽以满足海森堡不确定性关系。在时间演化过程中,压缩位置波包不会保持其形状,但在全周期后收缩回原始宽度之前,振荡周期的一半将变得更加宽。
动量波包相应地收缩或扩展,以便在任何时候不确定性最小\cite[]{Wallentowitz_Vogel_2002,Wallentowitz_Vogel_U_2002}。

\emph{压缩真空态(Squeezed vacuum state)}用公式\eqref{eq:superposition_states}的形式表述如下:
\begin{align}
    c_n=\begin{cases}
        \left(\frac{2\sqrt{\beta_s}}{\beta_s+1}\right)^{1/2} \left(\frac{\beta_s-1}{\beta_s+1}\right)^{n/2}(-1/2)^{n/2}\frac{\sqrt{n!}}{(n/2)!}e^{i n \phi}, &n\ even\\
        0, &n\ odd
    \end{cases}
\end{align}

参数$\beta_s$描述了状态压缩程度,压缩状态的位置方差在一定时间内减少:
\begin{align}
    \Delta x_s=\Delta x_0/\beta_s
\end{align}

\textcolor{red}{这里的 variance 应该是指的飘动范围还是方差?可称为:方差或者不确定度}

其中$\Delta x_0$是基态的方差。基态时$\beta_s=1$(因此名称为“压缩真空态”)。
对于$\beta_s>1$的的情况位置波函数比相干态情况时要窄,而对于$0<\beta_s<1$的情况动量波函数比相干态情况时要窄。
角度$\phi$描述了压缩状态相对于位置和动量方向的对齐,这可以在相空间中最好地可视化。
压缩状态的\emph{Wigner函数}具有椭圆等轮廓线\cite[]{Caldeira_Leggett_2002}。如果这些椭圆的主轴之一与位置坐标轴对齐,$\phi$等于零。压缩真空态Wigner函数的质心与相空间的原点重合。

压缩真空状态的概率分布$P_n$是独立的,这里局限于偶数状态:

\begin{align}
    P_n=\frac{2\sqrt{\beta_s}}{\beta_s+1} \left(\frac{\beta_s-1}{\beta_s+1}\right)^{n}(2)^{-n}\frac{n!}{[(n/2)!]^2},\ n\ even
\end{align}

对于强压缩,这个分布有一个持续到非常大$n$的拖尾;例如,对于 $\beta_s=40$,压缩真空的统计状态中有$16\%$处于$n=20$以上的状态。

挤压真空状态,如相干状态,具有非常紧凑的算子表示。它们是由操作员从基态生成的:
\begin{align}
    \hat{S}(\epsilon)=\exp\left\{\frac{1}{2}\epsilon^* \hat{C}_S(t)^2-\frac{1}{2}\epsilon [\hat{C}_S^\dagger(t)]^2\right\}
\end{align}

其中$\epsilon=r e^{i\phi}$,$r$是和$\beta_s$相关的,关系为$\beta_s=e^{2r}$。

\subsection[热分布]{热分布}

如果离子在高温$T$下与外部储层处于热平衡状态,则激发态$\ket{n}$的平均权重将与玻尔兹曼因子$\exp[-n\hbar\nu/(k_B T)]$成正比,其中$k_B$是玻尔兹曼常数。
当然,讨论单个离子的温度是没有意义的。然而,如果离子与储存器耦合,则多次测量数算子$\hat{N}$(确保每次测量后离子再平衡),可以根据许多不同的实现集合从平均结果$\bar{n}$中提取温度:
\begin{align}
    T=\frac{\hbar\nu}{k_B\ln\left(\frac{\bar{n}+1}{\bar{n}}\right)}
\end{align}

在考虑系综时,可以用密度矩阵来表征这个状态。
此外,本着选择具有最大\textcolor{red}{无知}(因此最大熵)的密度矩阵的精神,非对角元素必须为零。
这使得无法以如公式\eqref{eq:superposition_states}的形式写出热分布,使它相对应的密度矩阵在$T>0$时具有非零非对角元素。因此,即使文献中经常使用术语“热状态”,“热分布”似乎是对系综这种性质更合适的称呼。

经过一些简单的代数来归一化由玻尔兹曼因子加权的状态的轨迹后,密度矩阵可以写成:
\begin{align}
    \rho_{th}=\frac{1}{\bar{n}+1}\sum_{n=0}^{\infty}\left(\frac{\bar{n}}{\bar{n}+1}\right)^n\ket{n}\bra{n}
\end{align}

总体概率水平为:
\begin{align}
    P_n=\frac{\bar{n}^n}{(\bar{n}+1)^{n+1}}
\end{align}







\section[囚禁离子的光场耦合]{囚禁离子的光场耦合}
% \textcolor{red}{这部分主要参考文献\cite[p3-8]{Leibfried_Blatt_Monroe_Wineland_2003}}

在合适的电磁场的帮助下,囚禁离子的内部能级可以相互相干耦合,并且与离子的外部运动自由度耦合。
强力囚禁和耦合良好的离子可以等价于
\emph{Jaynes-Cummings 哈密顿量(Jaynes-Cummings Hamiltonian)}\cite[]{Janszky_Yushin_1986}
因此,许多致力于囚禁离子相干相互作用的工作都受到这种耦合在量子光学中所起的重要作用的启发。除了这种特殊情况之外,剩下的许多可能性会到多个运动量子之间的相互交换,类似于量子光学中的多光子跃迁。
此外,产生耦合的光场可以作为的能量来源,因此原子-光子耦合中隐含的能量守恒不必局限在囚禁离子的内部态和运动态之间的转换,也可以实现囚禁离子内部态的相互转换,比如吸收能量跃迁到更高的能级上。
最后,如果考虑了运动的全量子力学图,包括微运动引起的修正,则可能存在另一类跃迁,涉及在离子阱的RF电势中运动态整数倍数的相互转换或驱动场整数倍数的组合和长期运动(微运动边带)。

% \subsection[囚禁离子的光场耦合]{囚禁离子的光场耦合}
\subsection[二能级近似]{二能级近似\label{section:two_level_approximation}}
在常规的离子阱研究中,会把囚禁离子的电子能级结构近似为\emph{二能级系统(Two-level System)},这为研究提供了很大的方便。这个二能级系统表示为$\ket{g}$和$\ket{e}$,他们之间有着$\hbar \omega=\hbar(\omega_e-\omega_g)$的能量差。这对于实际的囚禁离子不总是适用的,仅在广场与离子两能级近似共振耦合且耦合的拉比频率远强于衰减到其它态的强度时才成立。不过这个条件对当今研究的多数实验系统中的离子(如镱离子、钡离子、钙离子等等)来说都是成立的。
相应的二能级哈密顿量$\hat{H}^{(e)}$的表述如下:
\begin{align}
    \hat{H}^{(e)}=\hbar(\omega_g\ket{g}\bra{g}+\omega_e\ket{e}\bra{e})\\
    =\hbar\frac{\omega_e+\omega_g}{2}(\ket{g}\bra{g}+\ket{e}\bra{e})\\
    +\hbar\frac{\omega}{2}(\ket{g}\bra{g}-\ket{e}\bra{e}) \label{eq:two_level_hamiltonian}
\end{align}

任何二能级系统有关的算子都可以被映射到$1/2$自旋算子基矢上,因此上述的$\hat{H}^{(e)}$及其相关的算子也可以被表示为公式\eqref{eq:pauli_matrices}中描述的泡利矩阵的形式,它们之间的映射关系如下:
\begin{align}
    \ket{g}\bra{g}+\ket{e}\bra{e}\mapsto \hat{I},\ \ket{g}\bra{e}+\ket{e}\bra{g}\mapsto \hat{\sigma}_x,\ \\
    i(\ket{g}\bra{e}-\ket{e}\bra{g})\mapsto \hat{\sigma}_y,\ \ket{e}\bra{e}-\ket{g}\bra{g}\mapsto \hat{\sigma}_z
\end{align}

在这种映射情况下,$\hat{H}^{(e)}$可以被表述为:
\begin{align}
    \hat{H}^{(e)}=\hbar\frac{\omega}{2}\sigma_z
\end{align}

相应的能量以$-\hbar(\omega_e+\omega_g)/2$重新缩放,以抑制公式\eqref{eq:two_level_hamiltonian}中与状态无关的能量贡献。

\subsection[耦合的理论表述]{耦合的理论表述}

为了以一种简单而充分的方式描述囚禁离子与光场的相互作用,如前一节所述,我们假设囚禁离子的运动在所有三个维度上都是谐波的。
下面的描述将包括囚禁电势的显式时间依赖,但在许多情况下,将离子的运动建模为三维静态势谐振子是足够的。
因为如果无量纲Paul阱参数$a_x$和$q_x^2$的模量相对于与静态势和射频势(见第\ref{section:ion_classical_motion}节)远小于1,则一般理论只会引入非常微小的变化。这对于实验中常用的离子阱是成立的。
内部状态和运动耦合的广义描述遵循文献\cite[]{Cirac_Garay_Blatt_Parkins_Zoller_2002,1996Paul}中的方法。

另外还假定,在光场的多极展开中处理最低阶展开就足够了,在所讨论的近共振电子状态之间产生一个不退化的矩阵元。
电子波函数的扩展远小于耦合场的波长这一事实证明了这一假设是合理的。
对于偶极允许跃迁,将用偶极近似来处理场,而对于偶极禁止跃迁,只考虑场的四极分量。
对于拉曼跃迁,近共振的中间能级将绝热消除,使这些跃迁在形式上等同于其它类型的跃迁。

\subsubsection[总哈密顿量和相互作用哈密顿量]{总哈密顿量和相互作用哈密顿量\label{section:total_hamiltonian}}
系统的总哈密顿量可以写作如下形式:
\begin{align}
    \hat{H}=\hat{H}^{(m)}+\hat{H}^{(e)}+\hat{H}^{(i)}
\end{align}

其中$\hat{H}^{(m)}$是沿着离子阱轴向的运动哈密顿量,如在第\ref{section:quantum_motion}节公式\eqref{eq:static_harmiltonian_oscillator}中讨论过的;
$\hat{H}^{(e)}$代表如第\ref{section:two_level_approximation}节中所述的离子的内部电子能级结构;
$\hat{H}^{(i)}$代表本部分将要讨论施加的光场与离子之间的耦合。

电偶极跃迁、电四极跃迁和激发拉曼跃迁可以在一个统一的框架中描述,该框架将某个共振拉比频率$\Omega$、有效光频率$\omega$和有效的波矢量$\mathbf{k}$与这些跃迁类型中的每一种相关联。
电偶极跃迁和电四级跃迁的耦合光场的频率和波矢量是相同的,但两者驱动受激拉曼跃迁的光场频率差$\omega=\omega_1-\omega_2$,波矢量差$\mathbf{k}=\mathbf{k_1}-\mathbf{k_2}$。

对于行波光场,所有三种跃迁类型都可以用以下形式的耦合哈密顿量来描述:
\begin{align}
    \hat{H}^{(i)}=(\hbar/2)\Omega(\ket{g}\bra{e}+\ket{e}\bra{g})\\
    \times\left[e^{i(k\hat{x}_S-\omega t + \phi)}+e^{-i(k\hat{x}_S-\omega t + \phi)}\right]
\end{align}

在$1/2$自旋代数中我们可以将其重新表述为:
\begin{align}
    \ket{e}\bra{g}\mapsto\hat{\sigma}_+=1/2(\hat{\sigma}_x+i\hat{\sigma}_y),\\
    \ket{g}\bra{e}\mapsto\hat{\sigma}_+=1/2(\hat{\sigma}_x-i\hat{\sigma}_y)
\end{align}

为了便于说明和理解,我们以一个维度的阐述为例。有效波矢量$\mathbf{k}$选为沿着离子阱中的$x$轴方向。转换到相互作用表象,可以得到自由哈密顿量$\hat{H}_0=hat{H}_{(m)}+hat{H}_{(e)}$与相互作用哈密顿量$\hat{V}=\hat{H}_{(i)}$的最简单的一个动力学图景。记$\hat{U}_0=\exp[-(i/\hbar)\hat{H}_0t]$,转换后的相互作用哈密顿量为:

\begin{align}
    \hat{H}_{int} &=\hat{U}_0^\dagger\hat{H}^{(i)}\hat{U}_0\\
    &=(\hbar/2)\omega e^{(i/\hbar)\hat{H}^{(e)}t}(\sigma_++\sigma_-)\\
    &\times e^{-(i/\hbar)\hat{H}^{(e)}t}e^{(i/\hbar)\hat{H}^{(m)}t}\left[e^{i(k\hat{x}-\omega t + \phi)}+e^{-i(k\hat{x}-\omega t + \phi)}\right]e^{-(i/\hbar)\hat{H}^{(m)}t}\\
    &=(\hbar/2)\Omega(\sigma_+e^{i\omega_0 t}+\sigma_-e^{-i\omega_0 t}) e^{(i/\hbar)\hat{H}^{(m)}t} \\
    &\left[e^{i(k\hat{x}-\omega t + \phi)}+e^{-i(k\hat{x}-\omega t + \phi)}\right]e^{-(i/\hbar)\hat{H}^{(m)}t}
\end{align}

上面公式表述中与时间相关的振动项提取出来后就是$\exp[\pm i (\omega\pm \omega_0)t]$。这两项一项振动频率为$\delta_f=\omega+\omega_0$,是快速振荡项;另一项振动为$\delta=\omega-\omega_0\ll \omega_0$,是慢速振荡项。在研究中我们一般会忽略快速振动项的贡献,也就是所谓的\emph{旋波近似(Rotating-wave Approximation)}。

引入\emph{Lamb-Dick参数(Lamb-Dick Parameter, LDP)}$\eta=kx_0$,其中$x_0=\sqrt{\hbar/(2m\nu)}$是参考振荡子基态波函数的$x$轴方向的扩展,海森堡图景下的$\hat{x}(t)$表述为:
\begin{align}
    k\hat{x}(t)=\eta\left\{\hat{a}u^*(t)+\hat{a}^\dagger u(t)\right\}
\end{align}

旋转波近似中的相互作用哈密顿量取其最终形式为:

\begin{align}
    \hat{H}_{int}(t)=(\hbar/2)\Omega\hat{\sigma}_+ \exp(i\{\phi+\eta[\hat{a}u^*(t)+\hat{a}^\dagger u(t)]-\delta t\})+H.c.
\end{align}

指数项中的时间依赖性由频率差$\delta$和$u(t)$控制。
考虑公式\eqref{eq:ut_expression}中的解形式及Lamb-Dicke 参数下的拓展有:
\begin{align}
    &\exp(i\{\phi + \eta [\hat{a}u^*(t)+\hat{a}^\dagger u(t)]-\delta t\})\\
    &=e^{i(\phi-\delta t)}\sum_{m=0}^{\infty} \frac{(i\eta)^m}{m!}\left\{\hat{a}e^{-i\beta_x\omega_{rf}t} \sum_{n=-\infty}^{\infty}C_{2n}^* \times e^{-i n\omega_{rf}t+H.c.}\right\}^m
\end{align}

很容易验证任何时刻的衰减满足下式:
\begin{align}
    (l'+l\beta_x)\omega_{rf}=\delta
\end{align}

其中$l$和$l'$为整数且$l\neq l', if l'\neq 0$,它们是哈密顿量中的两项慢速变化,贡献了含时变化的主要部分(其余部分可以忽略)。如果其中一个调制边带与静止离子的跃迁频率$\omega_0$重合,则该边带可以诱导离子内部状态跃迁。实际上,由于实验中$(|a_x|,q_x^2)\ll 1$,因此公式\eqref{eq:quantum_lowest_order_approximation}中的$\beta_x\omega_{rf}\approx\nu$且$C_o\approx(1+q_x/2)^{-1}$。于是相互作用哈密顿量就可以简化成如下形式:
\begin{align}
    \hat{H}_{int}(t)=(\hbar/2)\Omega_0\sigma_+ \exp\{i\eta(\hat{a}e^{-i\nu t}+\hat{a}^\dagger e^{i\nu t})\}e^{i(\phi-\delta t)}+ H.c. \label{eq:interaction_hamiltonian}
\end{align}

其中缩放的相互作用强度为$\Omega_0=\Omega/(1+q_x/2)$,这个缩放反映了射频驱动频率下波包\textcolor{red}{呼吸}引起的耦合减少。

\textcolor{red}{这里的波包 ‘breathing’ 应该怎么翻译?}

\subsubsection[拉比频率]{拉比频率\label{section:rabi_frequency}}

依赖失谐变量$\delta$,公式\eqref{eq:interaction_hamiltonian}中的哈密顿量会耦合一定的内态和运动态。如果公式\eqref{eq:interaction_hamiltonian}在$\eta$范围内,这会产生一个包含$\sigma_{\pm}$的组合项,它有$l$个$\hat{a}$算符和$m$个$\hat{a}^\dagger$算符并以频率$(l-m)\nu=s\nu$转动。如果$\delta\approx s\nu$的话,这些组合将是共振的,并将多态$\ket{g}\ket{n}$和$\ket{e}\ket{n+s}$耦合起来。对于$s>0(s<0)$的情况,耦合强度通常称为$|s|$级\emph{蓝(红)边带}拉比频率,表达为\cite[]{Leibfried_Meekhof_King_Monroe_Itano_Wineland_2002, Beige_Bose_Braun_Huelga_Knight_Plenio_Vedral_2000}:
\begin{align}
    \Omega_{n,n+s}=\Omega_{n+s,n}=\Omega_0|\braket{n+s|e^{i\eta(a+a^\dagger)}|n}|\\
    =\Omega_0 e^{-\eta^2/2}\eta^{|s|}\sqrt{\frac{n_<!}{n_>!}}L_{n_<}^{|s|}(\eta^2)
\end{align}

其中$n_<$是比$n+s$和$n$小,$n_>$是比$n+s$和$n$大,$L_n^\alpha(X)$是广义拉盖尔多项式:
\begin{align}
    L_n^\alpha(X)=\sum_{m=0}^{n}(-1)^m\begin{pmatrix}
        n+\alpha \\ n-m
    \end{pmatrix}\frac{X^m}{m!}
\end{align}

\subsection[总结说明]{总结说明}

上述的两部分(第\ref{section:total_hamiltonian}和\ref{section:rabi_frequency}节)介绍了光与离子耦合的主要基础。实际上关于光与离子的耦合还有许多内容可以介绍,这些耦合特性也在离子量子计算中被使用到了,不过我并不打算在这里就将其一一说明。作为替代,我将在后续结合离子量子计算的相关操作来对这些特性做出相应的介绍,比如离子量子比特的冷却(第\ref{section:yb_laser_cooling}节)、初始化(第\ref{section:yb_state_init}节)、探测(第\ref{section:yb_state_detection}节)、操作(第\ref{section:yb_state_manipulation}节)等。

\textcolor{red}{
这章剩下的离子本身部分考虑暂时不写,留到讲离子量子比特的时候写吧。}
% \subsection[离子内态测量]{离子内态测量}

% \subsection[离子运动态测量]{离子运动态测量}





% \section[离子的激光冷却]{离子的激光冷却}
% \textcolor{red}{主要介绍原理,不讲太多实验细节和结果,\textcolor{red}{这部分主要参考文献\cite[p16-19]{Leibfried_Blatt_Monroe_Wineland_2003}}}
% \subsection[多普勒冷却]{多普勒冷却}

% \subsection[边带冷却]{边带冷却}



% \section[Mølmer-Sørensen门]{Mølmer-Sørensen门}
% \textcolor{red}{这部分主要参考文献\cite[]{Azuma_2023}}





\section[系统的组成——离子阱系统]{系统的组成——离子阱系统}
\textcolor{red}{这部分尽量结合实验室的仪器设备展开叙述}

\subsection[囚禁电极]{囚禁电极}


\subsection[微波信号]{微波信号}


\subsection[真空系统]{真空系统}


\subsection[螺线管谐振腔]{螺线管谐振腔}











\section[系统的组成——光学系统]{系统的组成——光学系统}
\textcolor{red}{这部分尽量结合实验室的仪器设备展开叙述}

\subsection[冷却激光]{冷却激光}

\subsection[操控激光]{操控激光}






\section[系统的组成——测控系统]{系统的组成——测控系统}
\textcolor{red}{这部分尽量结合实验室的仪器设备展开叙述}
\subsection[测控系统的构架]{测控系统的构架}

