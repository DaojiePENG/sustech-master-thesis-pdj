% !TeX root = ../sustechthesis-example.tex

\chapter[量子计算]{量子计算}
\textcolor{red}{
这部分从经典计算讲到量子计算,并简单介绍当前常见的实现量子计算的平台... 
}
\section[量子计算基本原理介绍]{量子计算基本原理介绍}
\textcolor{red}{主要参考文献\cite[]{Williams2011}}
\subsection[比特]{比特}

\subsection[量子比特]{量子比特}

\subsection[量子比特门]{量子比特门}

\subsection[量子算法]{量子算法}









\section[量子计算的不同实现平台]{量子计算的不同实现平台}
\textcolor{red}{这部分需要分别找各个平台的文献进行整理,每个平台找一到两篇吧}
\subsection[离子量子计算]{离子量子计算}

\subsection[超导量子计算]{超导量子计算}

\subsection[原子量子计算]{原子量子计算}

\subsection[硅基量子计算]{硅基量子计算}

\subsection[光量子计算]{光量子计算}

\subsection[拓扑量子计算]{拓扑量子计算}






