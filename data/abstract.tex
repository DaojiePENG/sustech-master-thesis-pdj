% !TeX root = ../sustechthesis-example.tex

% 中英文摘要和关键字

\begin{abstract}
  量子计算因其潜在的应用前景而受到广泛关注强大的计算能力。在量子算法的配合下,量子计算机可以实现许多通过经典计算机难以实现的计算,如大素数分解、量子多体系统仿真等。

  离子阱量子计算是一种利用离子的两个稳定能级,作为量子比特“0”和“1”的状态,并利用激光或微波来控制能级之间的跃迁实现量子逻辑门的技术。离子阱量子计算机采用被“囚禁”在真空中的离子(带单个电荷的原子)作为量子比特,相比其他技术路线,离子阱量子计算具有相干时间长、保真度高、连通性好、编程方便等优点。

  量子测控系统是量子计算系统中的一个重要组成部分,它负责对量子比特进行精确的控制和测量,以实现量子计算的各种操作。在离子阱量子计算中,量子测控系统包括离子捕获、囚禁和冷却装置,激光器和微波源,探测器和放大器等。其中,离子捕获、囚禁和冷却装置用于捕获和囚禁离子,并将其冷却到极低的温度,以保证量子比特的相干性和稳定性;激光器和微波源用于对离子进行量子态的制备、演化和测量,实现量子逻辑门和量子算法;探测器和放大器用于探测离子的量子态,并将信号放大,以便进行后续处理。
  
  量子测控系统的设计需要综合诸多相关领域知识,如量子力学、光学、电子学等等。量子测控系统的性能直接影响着量子计算的准确性和可靠性,因此,如何提高量子测控系统的性能,是量子计算研究中的一个重要问题。

  % 关键词用“英文逗号”分隔,输出时会自动处理为正确的分隔符
  \thusetup{
    keywords = {量子计算, 离子阱, 测控系统, 电子学},
  }
\end{abstract}

\begin{abstract*}
  Quantum computing has attracted extensive attention due to its potentially powerful computing capabilities. With the cooperation of quantum algorithm,quantum computer can realize many calculations that are difficult to be realized by classical computers like factorization of large prime numbers, simulation of quantum many boy systems. 

  % Use comma as seperator when inputting
  \thusetup{
    keywords* = {Quantum Computation, Ion Trap, Measurement and Control System, Electonics},
  }
\end{abstract*}
