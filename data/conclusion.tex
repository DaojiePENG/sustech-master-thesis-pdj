% !TeX root = ../sustechthesis-example.tex

\begin{conclusion}

量子计算拥有着广阔的发展和应用前景。离子阱量子计算因其较高的相干时间和保真度是量子计算未来重要的发展方向之一。

本文从经典计算讲起,逐步过渡到量子计算的基础理论并重点阐述了离子阱量子计算体系中的关键概念。随后以镱离子量子计算平台为核心,讲述了离子量子比特的电离、冷却、初始化、探测、纠缠建立、比特控制等内容(第\ref{section:quantum_computation}、
% \ref{section:ion_trap_quantum_computation_system}、
\ref{section:yb_computation}章)。在第\ref{section:fpga_rtmq}章中介绍了一种基于FPGA实现的实时量子测控系统RTMQ,该测控系统相对于现有的量子测控系统所具有的显著优势是其严格的实时性和片上计算处理能力,具有广阔的应用前景。同时重点介绍了数字PID、数字滤波器等数字化系统模块的实现,为随后的各类系统控制奠定了基础。在第\ref{section:helical}章中介绍了离子阱系统中离子囚禁相关的一种关键的微波器件——螺线管谐振腔,通过基于有限元方法的仿真软件HFSS对谐振腔进行了建模和仿真,预实验结果取得了很好的一致性,同时建立了更为准确的谐振腔数学模型来描述谐振腔的特性。除此之外,也对谐振腔进行了优化设计,在谐振腔的稳定性、耦合方便性、模块化等方面有了很大的提升。接着给出了通过稳定谐振腔的输出来实现阱频率稳定的系统的搭建方法(第\ref{section:trap_frequency_stablization}章)。最后两章(第\ref{section:pulsed_laser_locking}、\ref{section:laser_power_locking}章)围绕离子比特的激光控制展开,基于RTMQ测控系统实现了脉冲激光拍频锁定、激光功率稳定等的系统搭建和测试,通过更多核心模块数字化的方式及大地提高了系统的灵活性和稳定性。

本文的重点在于量子比特的控制,以测控系统为中心主要关注离子比特控制系统中电子学和光学相关的子系统搭建问题。在上述测控系统、囚禁系统、光学稳定系统、光学操控系统的基础上可以进一步结合其它如离子电离、离子囚禁、离子冷却等整合实现离子量子比特门。虽然由于实验室整体方向规划和进度的原因,未能最终实现量子门操作,但在与门操作保真度直接相关的实验参数提升和稳定方面做了相当多的系统性工作,包括螺旋谐振腔设计方法的研究和改进以及阱频率、激光功率等参数的稳定方案,这些成果对于囚禁离子量子计算领域都具有普遍的实用意义。实际搭建中的脉冲光操控方案的离子阱系统如\ref{fig:ion_trap_system}图所示。进一步再结合光路寻址系统等可以实现多离子比特门的控制,最终面向通用量子计算计算机的实现。值得一提的是,芯片阱的出现使得离子阱的规模化、集成化和小型化的进程大大加快,是离子阱领域重要的研究方向之一,也是发展趋势所在,我们也正在开展这方面的探索和尝试。

% \textcolor{red}{最后再进行补充撰写\dots}


\end{conclusion}


